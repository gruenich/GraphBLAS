

\newpage
%%%%%%%%%%%%%%%%%%%%%%%%%%%%%%%%%%%%%%%%%%%%%%%%%%%%%%%%%%%%%%%%%%%%%%%%%%%%%%%%
\section{The SuiteSparse:GraphBLAS JIT} %%%%%%%%%%%%%%%%%%%%%%%%%%%%%%%%%%%%%%%%
%%%%%%%%%%%%%%%%%%%%%%%%%%%%%%%%%%%%%%%%%%%%%%%%%%%%%%%%%%%%%%%%%%%%%%%%%%%%%%%%
\label{jit}

SuiteSparse:GraphBLAS v8.0 adds a new JIT feature that greatly improves
performance of user-defined types and operators, and improves the performance
of built-in operators as well.  The JIT can compile kernels that are specific
to the matrix type and the operators that work on it.  In version v7.4.4 and
prior versions, user-defined types and operators were handled by {\em generic}
kernels that used function pointers for each operator and for any typecasting
required.  Even built-in types and operators were sometimes handled by the
generic kernels, if any typecasting was done, or if the specific operator,
monoid, or semiring was disabled when GraphBLAS was compiled.

\subsection{Using the JIT}

Using the JIT in a user application is simple:  by default, there is nothing
to do.  LAGraph v1.0.1 can use the JIT (and PreJIT) kernels without changing
a single line of code.\footnote{LAGraph v1.0.1 does not work with GraphBLAS
v8.0.0 but not because of the JIT.  LAGraph must be updated because it uses two
deprecated {\sf GxB\_SelectOp} operators; these are replaced with {\sf
GrB\_IndexUnaryOp} operators in the {\sf dev} branch of LAGraph, to appear
in a future release of LAGraph.}

Currently, the JIT compiles kernels for the CPU only, but a CUDA JIT is in
progress to exploit NVIDIA GPUs, in collaboration with Joe Eaton and
Corey Nolet, with NVIDIA.

When GraphBLAS is compiled, the \verb'cmake' build system creates a {\em cache}
folder where it will keep any kernels created and compiled by the JIT
(both source code and compiled libraries for each kernel).  The
default folder is \verb'~/.SuiteSparse/GrB8.0.0' for SuiteSparse:GraphBLAS
version v8.0.0, where the tilde refers to the user's home directory.
The version numbers in the folder name are set automatically, so that a new
version will ignore kernels compiled by an older version of GraphBLAS.  If the
\verb'GRAPHBLAS_CACHE_PATH' environment variable is set when GraphBLAS is
compiled, that variable defines the folder.  If the user's home directory
cannot be determined and the \verb'GRAPHBLAS_CACHE_PATH' environment variable
is not set, then JIT compilation is disabled and only PreJIT kernels can be
used.  The optional environment variable, \verb'GRAPHBLAS_CACHE_PATH', is also
read by \verb'GrB_init' when the user application runs.
See Section~\ref{cache_path} for a description of the valid characters
that can appear in the cache path.

When the user application starts, it can modify the location of the cache
folder after calling \verb'GrB_init'.  It can also modify the C compiler and
its flags, and can control when and how the JIT is used.
These changes are
made via \verb'GrB_set', and can be queried via
\verb'GrB_get'; refer to Section~\ref{options} for details, and
the \verb'GxB_JIT_*' settings:

\vspace{0.15in}
{\footnotesize
\begin{tabular}{lll}
\hline
field                       & value         & description \\
\hline
\verb'GxB_JIT_C_COMPILER_NAME' & \verb'char *' & C compiler for JIT kernels \\
\verb'GxB_JIT_C_COMPILER_FLAGS'& \verb'char *' & flags for the C compiler \\
\verb'GxB_JIT_C_LINKER_FLAGS' & \verb'char *' & link flags for the C compiler \\
\verb'GxB_JIT_C_LIBRARIES'    & \verb'char *' & libraries to link against (no cmake) \\
\verb'GxB_JIT_C_CMAKE_LIBS'   & \verb'char *' & libraries to link against (with cmake) \\
\verb'GxB_JIT_C_PREFACE'      & \verb'char *' & C code as preface to JIT kernels \\
\verb'GxB_JIT_C_CONTROL'      & see below     & CPU JIT control \\
\verb'GxB_JIT_USE_CMAKE'      & see below     & CPU JIT control \\
%\verb'GxB_JIT_ERROR_FALLBACK'& \verb'int32_t' & how to handle JIT compiler errors \\
\verb'GxB_JIT_ERROR_LOG'      & \verb'char *' & error log file \\
\verb'GxB_JIT_CACHE_PATH'     & \verb'char *' & folder with compiled kernels \\
\hline
\end{tabular}
}
\vspace{0.15in}

To control the JIT in the MATLAB \verb'@GrB' interface, use the \verb'GrB.jit'
method.  Refer to \verb'help GrB.jit' for details.

Kernels compiled during one run of a user application are kept in the cache
folder, so that when the user application runs again, the kernels do not have
to be compiled.  If the kernel relies on user-defined types and/or operators, a
check is made the first time the compiled kernel is loaded.  If the current
definition of the user-defined type or operator does not exactly match the
definition when the kernel was compiled, then the compiled kernel is discarded
and recompiled.  The stale kernel is overwritten with the new one, so there is
no need to for the user to take any action to delete the stale kernel from the
cache path.  If the cache path is changed via \verb'GrB_set', compiled kernels
in the old cache folder are not copied over.  New ones are compiled instead.

%----------------------------------------
\subsubsection{\sf GxB\_JIT\_C\_CONTROL}
%----------------------------------------

The usage of the CPU JIT can be controlled via \verb'GrB_get/set' using the
\verb'GxB_JIT_C_CONTROL' setting.  If the JIT is enabled at compile time, the
initial setting is \verb'GxB_JIT_ON'.  If the JIT is disabled at compile time
(by setting the cmake variable \verb'GRAPHBLAS_USE_JIT' to \verb'OFF'), the
initial setting is \verb'GxB_JIT_RUN', so that any PreJIT kernels can be run.
This setting can be modified; for example to disable the JIT and clear all
loaded JIT kernels from memory, use:

\begin{verbatim}
    GrB_set (GrB_GLOBAL, GxB_JIT_OFF, GxB_JIT_C_CONTROL) ;
\end{verbatim}

The above call to \verb'GrB_set' does not clear any PreJIT kernels, however,
since those are integral components of the single compiled GraphBLAS library
and cannot be cleared (see Section~\ref{prejit}).  It also does not clear any
compiled user functions, created by the JIT for \verb'GxB_*Op_new' when the
input function pointer is \verb'NULL'.

The following settings are available for \verb'GxB_JIT_C_CONTROL'.
For examples on how to use it, see
\verb'GraphBLAS/Demo/Program/gauss_demo.c'.

{\footnotesize
\begin{verbatim}
typedef enum
{
    GxB_JIT_OFF = 0,    // do not use the JIT: free all JIT kernels if loaded
    GxB_JIT_PAUSE = 1,  // do not run JIT kernels but keep any loaded
    GxB_JIT_RUN = 2,    // run JIT kernels if already loaded; no load/compile
    GxB_JIT_LOAD = 3,   // able to load and run JIT kernels; may not compile
    GxB_JIT_ON = 4,     // full JIT: able to compile, load, and run
}
GxB_JIT_Control ;
\end{verbatim} }

If the JIT is disabled at compile time via setting the \verb'GRAPHBLAS_USE_JIT'
option \verb'OFF', \verb'PreJIT' kernels are still available, and can be
controlled via the \verb'GxB_JIT_OFF', \verb'GxB_JIT_PAUSE', or
\verb'GxB_JIT_RUN' settings listed above.  If the application tries to set the
control to \verb'GxB_JIT_LOAD' or \verb'GxB_JIT_ON', the setting is changed to
\verb'GxB_JIT_RUN' instead.  This is not an error condition.  The resulting
setting can be queried via \verb'GrB_get', if desired.

If your copy of GraphBLAS has many PreJIT kernels compiled into it, or uses
many run-time JIT kernels, turning of the JIT with \verb'GxB_JIT_OFF' can be
costly.  This setting clears the entire JIT hash table.  Renabling the JIT and
using it will require the JIT table to be repopulated, including a check of
each PreJIT kernel the first time they are used.  If you wish to temporarily
disable the JIT, consider switching the JIT control to \verb'GxB_JIT_PAUSE' and
then back to \verb'GxB_JIT_RUN' to reenable the JIT.

%----------------------------------------
\subsubsection{JIT error handling}
%----------------------------------------

The JIT control setting can be changed by GraphBLAS itself, based on
following error conditions.  These changes affect all kernels, not just the
kernel causing the error.  If any of these cases occur, the call to GraphBLAS
returns \verb'GxB_JIT_ERROR', unless GraphBLAS runs out of memory, in which
case it returns \verb'GrB_OUT_OF_MEMORY' instead.  If the JIT is disabled
through any of these errors, it can be detected by \verb'GrB_get' to read the
\verb'GxB_JIT_C_CONTROL' state.

\begin{itemize}

\item When a kernel is loaded that relies on user-defined types and/or
operators, the definitions in the previously compiled kernel are checked
against the current definitions.  If they do not match, the old one is
discarded, and a new kernel will be compiled.  However, if the control is set
to \verb'GxB_JIT_LOAD', no new kernels may be compiled.  To avoid a continual
reloading and checking of stale kernels, the control is changed from
\verb'GxB_JIT_LOAD' to \verb'GxB_JIT_RUN'.  To solve this problem, delete the
compiled kernel with the stale definition, or enable the full JIT by setting
the control to \verb'GxB_JIT_ON' so that the kernel can recompiled with the
current definitions.

\item If a new kernel is to be compiled with the control set to
\verb'GxB_JIT_ON' but the source file cannot be created in the cache folder, or
a compiler error occurs, further compilation is disabled.  The control is
changed from \verb'GxB_JIT_ON' to \verb'GxB_JIT_LOAD'.
To solve this problem, make sure your application has write permission to the
cache path and that any user-defined types and operators are defined properly
so that no syntax error is detected by the compiler.

\item If a kernel is loaded but the lookup of the kernel function itself in the
compiled library fails, the control is changed to \verb'GxB_JIT_RUN' to prevent
this error from occuring again.  To solve this problem, delete the corrupted
compiled kernel from the cache folder.  This case is unlikely to occur since no
user action can trigger it.  It could indicate a system problem with loading
the kernel, or some kind of compiler error that allows the kernel to be
compiled but not loaded.

\item If an out-of-memory condition occurs in the JIT, the JIT control is
set to \verb'GxB_JIT_PAUSE'.  To solve this, try freeing up memory, use a
larger system, or solve smaller problems.

\end{itemize}

As a result of this automatic change in the JIT control setting, after the
first JIT error is returned, subsequent calls to GraphBLAS will likely succeed.
GraphBLAS will use a generic kernel instead.  To re-enable the JIT for
subsequent calls to GraphBLAS, the user application must reset the
\verb'GxB_JIT_C_CONTROL' back to \verb'GxB_JIT_ON'.

In many use cases of GraphBLAS (such as LAGraph), a function will create a type
or operator, use it, and then free it just before returning.  It would be far
to costly to clear the loaded kernel and reload it each time the LAGraph
function is called, so any kernels that use this type or operator are kept
loaded when the type or operator is freed.  The typical case is that when the
LAGraph function is called again, it will recreate the type or operator with
the identical name and definition.  The kernels that use these types or
operators will still be loaded and can thus be used with no overhead.

However, if a user-defined type or operator is freed and then redefined with
the same name but a different definition, any loaded kernels should be freed.
This case is not detected by GraphBLAS since it would be far too costly to
check each time a previously loaded kernel is called.  As a result, this
condition is only checked when the kernel is first loaded.  To avoid this
issue, if the user application frees a user-defined type or operator and
creates a new one with a different definition but with the same name, clear all
prior kernels by setting the control to \verb'GxB_JIT_OFF'.  Then turn the JIT
back on with \verb'GxB_JIT_ON'.  This clears all run-time JIT kernels so that
they will be checked when reloaded, and recompiled if their definitions
changed.  All PreJIT kernels are flagged as unchecked, just as they were
flagged by \verb'GrB_init', so that they will be checked the next time they
run.

%----------------------------------------
\subsubsection{\sf GxB\_JIT\_C\_COMPILER\_NAME}
%----------------------------------------

The \verb'GxB_JIT_C_COMPILER_NAME' string is the name of the C compiler to use,
or its full path.
By default it is set to the C compiler used to compile GraphBLAS itself.

%----------------------------------------
\subsubsection{\sf GxB\_JIT\_C\_COMPILER\_FLAGS}
%----------------------------------------

The \verb'GxB_JIT_C_COMPILER_FLAGS' string is the C compiler flags.
By default it is set to the C compiler flags used to compile GraphBLAS itself.

%----------------------------------------
\subsubsection{\sf GxB\_JIT\_C\_LINKER\_FLAGS}
%----------------------------------------

The \verb'GxB_JIT_C_LINKER_FLAGS' string only affects the kernel compilation
when cmake is not used to compile the kernels (see Section~\ref{use_cmake}).
By default it is set to the C link flags used to compile GraphBLAS itself.
If cmake is used to compile the kernels, then it determines the linker flags
itself, and this cannot be modified.

%----------------------------------------
\subsubsection{\sf GxB\_JIT\_C\_LIBRARIES}
%----------------------------------------

The \verb'GxB_JIT_C_LIBRARIES' string is used to set the libraries to link
against when cmake is not being used to compile the kernels (see
Section~\ref{use_cmake}).  For example, on Linux it is set by default to the
\verb'-lm', \verb'-ld', and OpenMP libraries used to link GraphBLAS itself.
Any standalone library name is prepended with \verb'-l'.  If cmake is used to
compile the kernels, this string is ignored.

%----------------------------------------
\subsubsection{\sf GxB\_JIT\_C\_CMAKE\_LIBS}
%----------------------------------------

The \verb'GxB_JIT_C_LIBRARIES' string is used to set the libraries to link
against when cmake is being used to compile the kernels (see
Section~\ref{use_cmake}).  For example, on Linux it is set by default to the
\verb'm', \verb'dl', and OpenMP libraries used to link GraphBLAS itself.
Libraries in the string should normally be separated by semicolons.  If cmake
is not used to compile the kernels, this string is ignored.

%----------------------------------------
\subsubsection{\sf GxB\_JIT\_C\_PREFACE}
%----------------------------------------

The \verb'GxB_JIT_C_PREFACE' string is added at the top of each JIT kernel.  It
is useful for providing additional \verb'#include' files that GraphBLAS does
not provide.  It can also be useful for diagnostics and for configuring the
\verb'PreJIT'.  For example, suppose you wish to tag specific kernels as having
been constructed for particular parts of an application.  The application can
modify this string to some unique comment, and then run some benchmarks that
call GraphBLAS.  Any JIT kernels created will be tagged with this unique
comment, which may be helpful to select specific kernels to copy into the
\verb'PreJIT' folder.

%----------------------------------------
\subsubsection{\sf GxB\_JIT\_USE\_CMAKE}
%----------------------------------------
\label{use_cmake}

Two methods are provided for compiling the JIT kernels: cmake, and a direct
compiler/link command.  On Windows, only cmake may be used, and this setting
is ignored (it is always true).  On Linux or Mac, the default is false since
a direct compile/link is faster.  However, it is possible that some compilers
are not handled properly with this method, so cmake can also be used on those
platforms by setting the value of \verb'GxB_JIT_USE_CMAKE' to true.

Normally the same version of cmake should be used to compile both GraphBLAS and
the JIT kernels.  However, compiling GraphBLAS itself requires cmake v3.16 or
later (v3.19 for some options), while compiling the JIT kernels only requires
cmake v3.13 or later.

%----------------------------------------
\subsubsection{\sf GxB\_JIT\_ERROR\_LOG}
%----------------------------------------

The \verb'GxB_JIT_ERROR_LOG' string is the filename of the optional error
log file.  By default, this string is empty, which means that any compiler
errors are routed to the \verb'stderr' output of the user process.  If set
to a non-empty string, any compiler errors are appended to this file.
The string may be \verb'NULL', which means the same as an empty string.

%----------------------------------------
\subsubsection{\sf GxB\_JIT\_CACHE\_PATH}
\label{cache_path}
%----------------------------------------

The \verb'GxB_JIT_CACHE_PATH' string is the full path to the user's cache
folder (described above).  The default on Linux/Mac is
\verb'~/.SuiteSparse/GrB8.0.0' for GraphBLAS version 8.0.0.  On Windows,
the cache folder is created inside the user's \verb'LOCALAPPDATA' folder,
called \verb'SuiteSparse/GrB8.0.0'.  When GraphBLAS starts,
\verb'GrB_init' checks if the \verb'GRAPHBLAS_CACHE_PATH' environment variable
exists, and initializes the cache path with that value instead of using the
default.

{\bf Restrictions:} the cache path is sanitized for security reasons.  No spaces
are permitted.  Backslashes are converted into forward slashes.  It can contain
only charactors in the following list:

\begin{verbatim}
        abcdefghijklmnopqrstuvwxyz
        ABCDEFGHIJKLMNOPQRSTUVWXYZ
        0123456789.-_/
\end{verbatim}

In addition, the second character in the string is allowed to be the colon
character (\verb':') to allow for the use of Windows drive letters.  Any
character outside of these rules is converted into an underscore (\verb'_').

%-------------------------------------------------------------------------------
\subsection{Compilation options: {\sf GRAPHBLAS\_USE\_JIT} and {\sf GRAPHBLAS\_COMPACT}}
%-------------------------------------------------------------------------------

The CPU JIT can be disabled at compile time by setting the
\verb'GRAPHBLAS_USE_JIT' option \verb'OFF' in the cmake build options.  Good
performance will be obtained only by using the \verb'FactoryKernels' or the
\verb'PreJIT' kernels that are compiled into GraphBLAS when it is first
compiled with \verb'cmake'.  By default, \verb'GRAPHBLAS_USE_JIT' is \verb'ON',
to enable the CPU JIT.

With the introduction of the JIT kernels, it is now possible to obtain good
performance in GraphBLAS without compiling the many {\em factory kernels} that
appear in the \verb'GraphBLAS/Source/FactoryKernels' directory.  If the JIT is
enabled, GraphBLAS will still be fast, once the JIT kernels are compiled, or by
using any \verb'PreJIT' kernels.  To compile GraphBLAS without its
\verb'FactoryKernels', enable the \verb'COMPACT' option in the cmake build
options.  By default, \verb'COMPACT' is off, to enable the
\verb'FactoryKernels'.

When GraphBLAS is compiled with \verb'GRAPHBLAS_USE_JIT' set to \verb'OFF', the
\verb'GxB_JIT_C_CONTROL' may be set to \verb'GxB_JIT_OFF',
\verb'GxB_JIT_PAUSE', or \verb'GxB_JIT_RUN'.  No kernels will be loaded at
run-time (the \verb'GxB_JIT_LOAD' setting is disabled and treated as
\verb'GxB_JIT_RUN'), and no new kernels will be compiled at run-time (the
\verb'GxB_JIT_ON' is disabled and treated as \verb'GxB_JIT_RUN').  Only
pre-existing \verb'PreJIT' kernels can be run, described in
Section~\ref{prejit}.

If both \verb'GRAPHBLAS_USE_JIT' is set \verb'OFF' and
\verb'GRAPHBLAS_COMPACT' is set \verb'ON', all features of GraphBLAS will be
functional.  The only fast kernels available will be the \verb'PreJIT' kernels
(if any).  Otherwise, generic kernels will be used, in which every single
operator is implemented with a function pointer, and every scalar assignment
requires a \verb'memcpy'.  Generic kernels are slow, so using this combination
of options is not recommended when preparing GraphBLAS for production use,
benchmarking, or for a Linux distro or other widely-used distribution, unless
you are able to run your application in advance and create all the JIT kernels
you need, and then copy them into \verb'GraphBLAS/PreJIT'.  This would be
impossible to do for a general-purpose case such as a Linux distro, but
feasible for a more targetted application such as FalkorDB.

%-------------------------------------------------------------------------------
\subsection{Adding {\sf PreJIT} kernels to GraphBLAS}
%-------------------------------------------------------------------------------
\label{prejit}

When GraphBLAS runs, it constructs JIT kernels in the user's cache folder,
which by default is \verb'~/.SuiteSparse/GrB8.0.0' for v8.0.0.  The
kernels placed in a subfolder (\verb'c') and inside that folder they are
further subdivided arbitrarily into subfolders (via an arbitary hash).  The
files are split into subfolders because a single folder may grow too large for
efficient access.  Once GraphBLAS has generated some kernels, some or all of
them kernels can then incorporated into the compiled GraphBLAS library by
copying them into the \verb'GraphBLAS/PreJIT' folder.  Be sure to move any
\verb'*.c' files into the single \verb'GraphBLAS/PreJIT' folder; do not keep
the subfolder structure.

If GraphBLAS is then recompiled via cmake, the build system will detect these
kernels, compile them, and make them available as pre-compiled JIT kernels.
The kernels are no longer ``Just-In-Time'' kernels since they are not compiled
at run-time.  They are refered to as \verb'PreJIT' kernels since they were at
one time created at run-time by the GraphBLAS JIT, but are now compiled into
GraphBLAS before it runs.

{\bf It's that simple.}  Just copy the source files for any kernels you want
from your cache folder (typically \verb'~/.SuiteSparse/GrB8.0.0/c') into
\verb'GraphBLAS/PreJIT', and recompile GraphBLAS.  There's no need to change
any other cmake setting, and no need to do anything different in any
applications that use GraphBLAS.  Do not copy the compiled libraries; they are
not needed and will be ignored.  Just copy the \verb'*.c' files.

If the resulting GraphBLAS library is installed for system-wide usage (say in a
Linux distro, Python, RedisGraph, etc), the \verb'GraphBLAS/PreJIT' kernels
will be available to all users of that library.  They are not disabled by the
\verb'GRAPHBLAS_USE_JIT' option.

Once these kernels are moved to \verb'GraphBLAS/PreJIT' and GraphBLAS is
recompiled, they can be deleted from the cache folder.  However, even if they
are left there, they will not be used since GraphBLAS will find these kernels
as PreJIT kernels inside the compiled library itself (\verb'libgraphblas.so' on
Linux, \verb'libgraphblas.dylib' on the Mac).  GraphBLAS will not be any slower
if these kernels are left in the cache folder, and the compiled library size
will not be affected.

If the GraphBLAS version is changed at all (even in the last digit), all
\verb'GB_jit_*.c' files in the \verb'GraphBLAS/PreJIT' folder should be
deleted.  The version mismatch will be detected during the call to
\verb'GrB_init', and any stale kernels will be safely ignored.  Likewise, if a
user-defined type or operator is changed, the relevant kernels should also be
deleted from \verb'GraphBLAS/PreJIT'.  For example, the
\verb'GraphBLAS/Demo/Program/gauss_demo.c' program creates a user-defined
\verb'gauss' type, and two operators, \verb'addgauss' and \verb'multgauss'.  It
then intentionally changes one of the operators just to test this feature.  If
the type and/or operators are changed, then the \verb'*gauss*.c' files in the
\verb'GraphBLAS/PreJIT' folder should be deleted.

GraphBLAS will safely detect any stale \verb'PreJIT' kernels by checking them
the first time they are run after calling \verb'GrB_init' and will not use them
if they are found to be stale.  If the JIT control is set to \verb'GxB_JIT_OFF'
all PreJIT kernels are flagged as unchecked.  If the JIT is then renabled by
setting the control to \verb'GxB_JIT_RUN' or \verb'GxB_JIT_ON', all PreJIT
kernels will be checked again and any stale kernels will be detected.

If a stale PreJIT kernel is found, GraphBLAS will use its run-time JIT to
compile new ones with the current definitions, or it will punt to a generic
kernel if JIT compilation is disabled.  GraphBLAS will be functional, and fast
if it can rely on a JIT kernel, but the unusable stale PreJIT kernels take up
space inside the compiled GraphBLAS library.  The best practice is to delete
any stale kernels from the \verb'GraphBLAS/PreJIT' folder, or replace them with
newly compiled JIT kernels from the cache folder, and recompile GraphBLAS.

It is safe to copy only a subset of the JIT kernels from the cache folder into
\verb'GraphBLAS/PreJIT'.  You may also delete any files in
\verb'GraphBLAS/PreJIT' and recompile GraphBLAS without those kernels.  If
GraphBLAS encounters a need for a particular kernel that has been removed from
\verb'GraphBLAS/PreJIT', it will create it at run-time via the JIT, if
permitted.  If not permitted, by either compiling GraphBLAS with the
\verb'GRAPHBLAS_USE_JIT' option set ot \verb'OFF', or by using
\verb'GxB_JIT_C_CONTROL' at run-time, the factory kernel or generic kernel will
be used instead.  The generic kernel will be slower than the PreJIT or JIT
kernel, but GraphBLAS will still be functional.

In addition to a single \verb'README.txt' file, the \verb'GraphBLAS/PreJIT'
folder includes a \verb'.gitignore' file that prevents any files in the folder
from being synced via \verb'git'.  If you wish to add your PreJIT kernels to a
fork of GraphBLAS, you will need to revise this \verb'.gitignore' file.

%-------------------------------------------------------------------------------
\subsection{{\sf JIT} and {\sf PreJIT} performance considerations}
%-------------------------------------------------------------------------------
\label{jit_performance}

To create a good set of PreJIT kernels for a particular user application, it is
necessary to run the application with many different kinds of workloads.  Each
JIT or PreJIT kernel is specialized to the particular matrix format, data type,
operators, and descriptors of its inputs.  GraphBLAS can change a matrix format
(from sparse to hypersparse, for example), at its discretion, thus triggering
the use of a different kernel.  Some GraphBLAS methods use heuristics to select
between different methods based upon the sparsity structure or estimates of the
kind or amount of work required.  In these cases, entirely different kernels
will be compiled.  As a result, it's very difficult to predict which kernels
GraphBLAS will find the need to compile, and thus a wide set of test cases
should be used in an application to allow GraphBLAS to generate as many kernels
as could be expected to appear in production use.

GraphBLAS can encounter very small matrices, and it will often select its
bitmap format to store them.  This change of format will trigger a different
kernel than the sparse or hypersparse cases.  There are many other cases like
that where specific kernels are only needed for small problems.  In this case,
compiling an entirely new kernel is costly, since using a compiled kernel will
be no faster than the generic kernel.  When benchmarking an application to
allow GraphBLAS to compile its JIT kernels, it may be useful to pause the JIT
via \verb'GxB_JIT_PAUSE', \verb'GxB_JIT_RUN', or \verb'GxB_JIT_LOAD', when the
application knows it is calling GraphBLAS for tiny problems.  These three
settings keep any loaded JIT kernels in memory, but pauses the compilation of
any new JIT kernels.  Then the control can be reset to \verb'GxB_JIT_ON' once
the application finishes with its tiny problems and moves to larger ones where
the JIT will improve performance.  A future version of GraphBLAS may allow
this heuristic to be implemented inside GraphBLAS itself, but for now, the
JIT does not second guess the user application; if it wants a new kernel,
the JIT will compile it if the control is set to \verb'GxB_JIT_ON'.

%-------------------------------------------------------------------------------
\subsection{Mixing JIT kernels: MATLAB and Apple Silicon}
%-------------------------------------------------------------------------------

In general, the JIT kernels compiled by the C interface and the kernels
compiled while using GraphBLAS in MATLAB are interchangable, and the same cache
folder can be used for both.  This is the default.

However, when using the \verb'@GrB' MATLAB interface to GraphBLAS on Apple
Silicon, using an older version of MATLAB, the MATLAB JIT kernels are compiled
as x86 binaries and executed inside MATLAB via Rosetta.  The pure C
installation may compile native Arm64 binaries for its JIT kernels.  Do not mix
the two.  In this case, set another cache path for MATLAB using \verb'GrB.jit'
in MATLAB, or using \verb'GrB_set' in the C interface for your native Arm64
binaries.

This issue does not apply to for recent MATLAB versions on the Mac,
which are native.

%-------------------------------------------------------------------------------
\subsection{Updating the JIT when GraphBLAS source code changes}
%-------------------------------------------------------------------------------

If you edit the GraphBLAS source code itself or add any files to
\verb'GraphBLAS/PreJIT', read the instructions in
\verb'GraphBLAS/JITpackage/README.txt' for details on how to update the JIT
source code.

If your cache folder (\verb'~/.SuiteSparse/GrBx.y.z') changes in any way
except via GraphBLAS itself, simply delete your cache folder.  GraphBLAS will
then reconstruct the kernels there as needed.

%-------------------------------------------------------------------------------
\subsection{Future plans for the {\sf JIT} and {\sf PreJIT}}
%-------------------------------------------------------------------------------
\label{jit_future}

\subsubsection{Kernel fusion}
The introduction of the JIT and its related PreJIT kernels allow for the future
exploitation of kernel fusion via an aggressive exploitation of the GraphBLAS
non-blocking mode.  In that mode, multiple calls to GraphBLAS can be fused into
a single kernel.  There are far to many possible variants to allow a fused
kernel to appear in the \verb'GraphBLAS/Source/FactoryKernels' folder, but
specific fused kernels could be created by the JIT.

\subsubsection{Heuristics for controlling the JIT}
As mentioned in Section~\ref{jit_performance}, GraphBLAS may compile JIT
kernels that are used for only tiny problems where the compile time of a single
kernel will dominate any performance gains from using the compiled kernel.  A
heuristic could be introduced so that it compiles them only for larger
problems.  The possible downside of this approach is that the same JIT kernels
might be needed later for larger problems.

\subsubsection{CUDA / SYCL / OpenCL kernels}
The CUDA JIT will enable NVIDIA GPUs to be exploited.  There are simply too
many kernels to create at compile time as the ``factory kernels.''  This CUDA
JIT is in progress.  A related JIT for SYCL / OpenCL kernels is under
consideration.

\subsubsection{Better performance for multithreaded user programs:}
This version is thread-safe when used in a multithread user application, but a
better JIT critical section (many readers, one writer) might be needed.  The
current critical section may be sufficiently fast since the typical case of
work done inside the critical section is a single hash table lookup.  However,
the performance issues related to this have not been tested.  This has no
effect if all parallelism is exploited only within GraphBLAS.  It only
affects the case when multiple user threads each call GraphBLAS in parallel
(using the \verb'GxB_Context'; see Section~\ref{context}).

